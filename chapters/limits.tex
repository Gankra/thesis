\chapter{Limitations of Ownership}
\label{ch:limits}


We've seen that ownership allows us to cleanly model all of the major classes
of error that we were interested in solving. This is because ownership's primary
role is to prevent data from being accessed at inappropriate times, and this is
what all of those problems boiled down to. However ownership is unable to
properly model some problems and can severely limit some designs.





\section{Cyclic References}

Ownership is severely biased towards unique ownership. This implies that data
is laid out in a tree-like manner, without any cycles. If one's problem can
be modeled without referential cycles, then ownership is pleasant and easy to
use. Otherwise, ownership becomes quite burdensome.

Any naive attempt to create a graph, tree with parent pointers, doubly-linked
list, or a value that contains pointers to itself will quickly lead to the
compiler shutting everything down. This is necessitated by the fact that Rust
allows memory to be manually managed, but demands that pointers can never dangle.
Anyone who has tried to build a pointer-based data structure knows all too well
how easy it is to forget to update some of the pointers after a node is removed.

Therefore Rust refusing to compile this sort of code is \emph{correct}. Pointer-based
data structures are wildly unsafe without garbage collection, and are error-prone
even \emph{with} garbage collection. Still, they're useful, and people want to write
them. Questions about this problem are so frequent that we wrote an entire book
on the topic \cite{too-many-lists}.

The most straight-forward solution to this problem is to embrace garbage collection.
Although Rust doesn't provide any serious automatic tracing collector, it does
provide manual reference counted pointers (Rc). Rc is Rust's primary mechanism for
\emph{shared ownership}. No Rc is the true owner of the pointed-to data. Instead,
all Rcs have equal ownership. Because the data pointed to by an Rc is always
potentially shared, only immutable shared access is provided. On its own,
this significantly limits the usefulness of Rc -- it's not possible to construct
a cycle with strict evaluation and immutability!

The solution to this problem is interior mutability. With interior mutability
and reference counting, one can safely build and manage cyclic data structures
at the cost of reference counting and runtime ownership checking. This also
comes at a significant ergonomic cost. Compare these equivalent Rust and
Java programs, which just make a cycle of two nodes:

\begin{minted}{rust}
use std::rc::Rc;
use std::cell::RefCell;

// Magical annotation to get a Node::default constructor
#[derive(Default)]
struct Node {
    data: u32,
    next: Option<Rc<RefCell<Node>>>,
    prev: Option<Rc<RefCell<Node>>>,
}

fn main() {
    let a = Rc::new(RefCell::new(Node::default()));

    let b = Rc::new(RefCell::new(Node {
        data: 1,
        next: Some(a.clone()),
        prev: Some(a.clone()),
    }));

    a.borrow_mut().next = Some(b.clone());
    a.borrow_mut().prev = Some(b.clone());
}
\end{minted}

\begin{minted}{java}
class Main {
    public static void main (String[] args) {
        class Node {
            int data;
            Node next;
            Node prev;
        }

        Node a = new Node();

        Node b = new Node();
        b.data = 1;
        b.next = a;
        b.prev = a;

        a.next = b;
        a.prev = b;
    }
}
\end{minted}

The Rust program is completely bogged down in boiler-plate to properly create
the pointers, copy them, and update the contents. This is to some extent desirable
as Rust likes to make costs explicit, and node-based data structures are quite
heavyweight for low level programming. Defaulting to pervasive garbage collection
and nullable everything ends up being quite nice for this use case.

The primary workaround for this problem is to use arrays and indices instead of
pointers. For many workloads, this is completely sufficient, and even more
efficient than a naive graph. As a rough example:

\begin{minted}{rust}
#[derive(Default)]
struct Node {
    data: u32,
    next: Option<usize>,
    prev: Option<usize>,
}

fn main() {
    let mut graph = Vec::new();
    graph.push(Node::default());
    graph.push(Node {
        data: 1,
        next: Some(0),
        prev: Some(0),
    });

    graph[0].next = Some(1);
    graph[0].prev = Some(1);
}
\end{minted}

This is, in essence, completely abandoning the borrow checker. It only reasons
about pointers, and we're using indices into the array. This allows us to
completely invalidate nodes by removing the things they ``point'' to, and even
perform what is logically equivalent to a use-after free by removing a node
and then reinserting a node. However we're unable to violate
memory safety in this manner, because the indices will be bounds checked.
We'll just get nonsensical results, which is of course quite easy to do with
any node-based structure.

The petgraph library \cite{petgraph} provides a more complete interface for
building and working with graphs in Rust. Internally, it just boils down to
this basic design. Graphs are an array of nodes and an array of edges,
and nodes and edges just store indices into those arrays. This structure is
used by the Rust compiler itself (which is written entirely in Rust).

Of course, this is only necessary if one wants to implement cyclic structures
safely. It's entirely possible to implement them exactly as they would be in C
in all their unsafe glory. This is how Rust's LinkedList and BTreeMap
collections are implemented. As usual, the unsafety is easily hidden behind
a reasonable and efficient public interface.





\section{Leaks}

Leaks aren't well-modeled by Rust's ownership system because it's based on affine
types, which allow us to forget about values at any time.
In order to encode that a value must be properly consumed, we need to be able
to express that a value must be used \emph{exactly} once. Types with this property
are said to be \emph{linear}.

A poor-man's linear-typing can be acquired with \emph{destructors}. A value's destructor
is some code to execute when it goes out of scope. This effectively mandates that
values with destructors be used exactly once: the owner who decides to drop the
value on the ground is forced to invoke the value's destructor.

The most obvious limitation to this approach is that since destructors are implicitly invoked,
they cannot take any additional context. For instance, it may be desirable
to have a type that is allocated using a custom allocator, but does not store
a pointer to that allocator. The type therefore has insufficient information
to clean itself up, requiring the allocator be passed to it. A more robust linearity
system would allow us to encode this by requiring the destructor be explicitly invoked with
the allocator as an argument. Destructors also can't return anything to indicate
success or failure, which is problematic for the automatic closing of files.

That said, destructors do a great job for many types. Collections and connections
have a natural final operation that requires no additional context.
Collections free their allocations and connections close themselves.
Destructors also have the distinct advantage over more general linearity in that
they compose well with generic code. If some generic code wants to drop a value
on the ground, it can always do this knowing any linear requirements will be
satisfied by the destructor.

Unfortunately, ensuring that destructors execute is a nasty problem.
At the limit, hardware can fail and programs can abort. We just need to live
with that fact. For most resources this is actually fine; either the resources are
rendered irrelevant by the program ending, or the operating system will automatically
clean them all up itself.

Even accepting those circumstances, strict
linearity can be burdensome. In particular, it is desirable to be able to
create reference-counted cycles of types that have destructors. If all references to
such a cycle are lost it will keep itself alive, leaking the destructors it
owns. Further, it is sometimes desirable to manually prevent a destructor
from running, particularly when decomposing a value into its constituent parts. For
instance, one may wish to downgrade a pointer that would normally free its allocation
to a raw pointer, perhaps to pass to a C API.

There are various solutions to this problem, but most languages that include
destructors (C++ C\#, Java, D, ...) generally accept that sometimes
they won't run. They're convenient and generally
reliable, but if one \emph{really} needs something to happen, they can't always be
relied on. Rust also takes this stance. In particular, one can't pass a value with
a destructor to an arbitrary third-party and rely on the destructor to be called,
even if the compiler believes any borrows they held have expired.

It should however be noted that this is \emph{completely} a library decision,
and not a language one. Rust's standard library decided that leaking destructors
was a safe operation, and safe interfaces need to assume it can happen. However
there exists an alternative standard library that adopted a different stance,
making this an unsafe operation \cite{rust-fork}. We won't dwell on the details,
but the basic idea is to encode leak-safety as an unsafe trait like the standard
library does for thread-safety.

As a concrete example of dealing with the ability to have a destructor leaked,
we can consider Rust's \emph{drain} interface. Drain is a
utility for efficiently performing bulk removal from the middle of a growable
array. Arrays require their elements to be stored contiguously, so removing
from the middle of one generally requires the elements after it to be
shifted back to fill the hole. Doing this repeatedly is incredibly expensive,
so it's desirable to be able to defer the shift until we are done removing
elements. Doing this means the array is in an unsound state while the
removals are happening, and we don't want this to be observable by the client.

One solution to this problem is to require the caller to pass in a
target buffer for all the elements that will be removed. Then `drain` can
just run to completion without ever yielding control to the client. Unfortunately,
this is quite inflexible.

To get more control we once again created an external interface, implementing
`drain` as an iterator. Elements are removed one at a time, allowing the user to
decide what is done with them without any need for allocating a buffer. At first
blush, the safety of this interface seems trivially solved by ownership. Drain is mutating the
array, so it contains a mutable reference to the array. That means the array is
inaccessible through any interface but the iterator while it exists. The
shifting can then be done in the iterator's destructor, which is exactly when
the borrow it holds expires. Perfect!

\begin{minted}{rust}
// A growable array with 5 numbers
let mut arr = vec![0, 1, 2, 3, 4];

// Drain out elements 1 to 3 (exclusive).
// The result of `drain` is an iterator
// that we pass to the for loop to drive.
for x in arr.drain(1..3) {
    // arr is statically inaccessible here
    println!("{}", x);
}
// backshifting is performed here
// arr is now accessible again

// 1 and 2 are now gone
assert_eq!(arr, vec![0, 3, 4]);
\end{minted}

Unfortunately, the soundness of this design relies on the user of our interface
allowing the destructor to run. No reasonable code would ever \emph{not} run the
destructor, but since one can violate memory safety by accessing these empty
indices, we cannot tolerate the possibility.

Thankfully, all is not lost. The solution to our problem is in fact quite simple.
Rather than relying on the destructor to put the array in a sound state,
we will \emph{start} the drain by putting the array into an incorrect but otherwise
sound state, and rely on the destructor to put the array into the correct state.
If the user of our interface prevents the destructor from running,
they'll get the incorrect array which will probably cause their program to behave
incorrectly, but be unable to violate memory safety. Since no reasonable program
will ever trigger this, the possibility of incorrect state isn't particularly
worrying.

In the case of `drain`, the incorrect state is setting the length of the array to be zero.
The destructor in turn just sets the length to the correct value (which it would
have done anyway). So the overhead for this scheme is zeroing out a single
integer, which is completely negligible considering the operation we're performing.
The consequence of this is that all the elements in the array will be
lost if the destructor is leaked. There's a certain fairness to this:
\emph{you leak me, I leak you!} We call this \emph{leak amplification}.

Not all interfaces have such a convenient state though. If this is the case, one
may instead ensure some code executes by using an internal interface. Rather
than returning an object that can be forgotten, we can require the user to pass a
function that will be passed a \emph{pointer} to the object, preventing them from
gaining true ownership of it. We can then guarantee that the object's
destructor always runs by dropping it in our own function. In fact, because we
always control the value, we don't even need to use a destructor to ensure the
cleanup code runs. We can just have that code at the end of the function.





\section{Scoped Threads}

With that said, we finally have all the tools to understand the interface we
demonstrated at the start of this thesis. Scoped threads allow data to be borrowed
by another thread, and even mutated. As we discussed previously, this is quite
dangerous, because borrows don't understand concurrency. So how does this work?
The initial scoped thread interface was designed to work as follows:

\begin{minted}{rust}
use std::thread;

fn main() {
    // An array of integers, stored on the stack
    let mut data = [1, 2, 3];

    {
        // An array of thread join guards. We will destroy this
        // when we want to block on the threads completing.
        let mut guards = Vec::new();

        // Get pointers to each element in the array
        for x in &mut array {
            // Spawn a thread to increment this element of the array
            let guard = thread::scoped(move || {
                *x += 1;
            });
            guards.push(guard);
        }

        // guards go out of scope here, so we block on all the child
        // threads joining.
    }

    // Array is now done being modified, and safe to read
    println!("{:?}", array);
}
\end{minted}

The interesting piece of this API was the guards that were returned
by the scoped function. Each guard claimed to store the closure the
scoped function was passed. This in effect made them borrow all of the
data that the closure could access. So as long as the guard existed, nothing
could interfere with data that the spawned threads were accessing. No magic
here, just the borrow checker doing its job.

However the borrows expire when the guard goes away, so the guard needs to
make sure that the spawned thread is no longer running. The only way to do
that is to wait for the thread to finish, and that's exactly what a guard's
destructor did. But we have a problem: leaking the destructor. If the destructor
is prevented from running, the borrow checker will believe all borrows have
expired, and allow the array to be accessed while it's still potentially
being mutated on another thread.

Unfortunately for the scoped interface, there's no valid leak amplification strategy.
We would have to somehow invalidate all the data that was borrowed, but that's
not possible. So we must use an internal interface, resulting in the example
we saw in the introduction:

\begin{minted}{rust}
[dependencies]
crossbeam = 0.1.6

-----

extern crate crossbeam;

fn main() {
    let mut array = [1, 2, 3];

    // scope is essentially an `&mut Vec<Guard>`
    crossbeam::scope(|scope| {
        for x in &mut array {
            // scope.spawn is essentially guards.push(guard)
            scope.spawn(move || {
                *x += 1;
            });
        }
    });

    println!("{:?}", array);
}
\end{minted}

This design is a bit less flexible. In particular, one loses control of how
the guards are stored and when they are destroyed. A more complicated design
could in principle expose that functionality, but we haven't yet found the need
to do so.

Regardless, the end result is impressive: in spite of the limitations of ownership
being a fairly simple and limited system, we can use it to safely share data stored
on the stack of one thread with several other threads without unnecessary
synchronization!

