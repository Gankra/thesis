\chapter{Conclusion}
\label{ch:outro}

We have demonstrated how several classic safety and correctness issues can
be reduced to an issue of trust between code and the data it shares. Of particular
interest are the strategies we have identified for managing this trust:
naivety, paranoia, and suspicion. These strategies respectively correspond to the major
competing concerns of control, safety, and ergonomics in interface design.

Rust's model of ownership enables each of these strategies to
be used together. At the lowest level, Unsafe Rust provides naive unsafe interfaces
as simple building blocks. Privacy enables developers to build on top of these
unsafe interfaces, exposing paranoid or suspicious interfaces as needed. Affine
typing enables developers to rely on the fact that data isn't shared to safely
mutate or invalidate it. Region analysis enables developers to share data
safely knowing that it won't be mutated or stored longer than its lifetime.
Of particular note is the philosophy of mutability and sharing being exclusive,
which trivially generalizes to data race freedom.

We've seen how this comes together in several interface designs. Iterators
provide an interface for safely indexing into an array without any
bounds or consistency checks because ownership prevents their invalidation.
The entry API demonstrate how algorithmic state can be captured and exposed externally
to make more flexible interfaces. Ownership can be pushed to extreme limits to
with generativity, enabling interfaces to statically require that individual
values have the same origin.

Other interfaces demonstrate the limits of ownership. Structures with cyclic
references like graphs are hostile towards ownership, requiring region analysis
to be bypassed in some way. The drain API demonstrates how ownership is poor
at guaranteeing that good things \emph{are} done, rather than guaranteeing that
bad things aren't.

All of this comes together to produce an impressive feat: the ability to have
threads safely access data stored on the stack of other threads without
unnecessary synchronization.

Due to the youth of Rust, there is much open exploration to be done. This
thesis only observes ownership intuitively and informally.
Rigorously demonstrating the soundness of Rust and ownership still needs to
be done. Since Rust is under active development, this is a bit of a moving
target.

It's unclear what extensions to ownership can be soundly added in the future.
Of particular interest is what extensions are unsound to \emph{ever} add given the
interfaces that are implemented in Rust today. Proper linear types are a
frequently requested extension whose interactions haven't been fully explored.
It's also unclear what the consequences of more powerful region
analysis would be. Since region analysis is actually expected to be revamped
withing the year, this is an incredibly relevant question.

Part of the issue is that much of Rust's semantics are actually emergent from
library decisions. The entire notion of thread-safety in Rust is implemented as
a library! As a result, deciding that some operation is safe can mandate that other
operations are unsafe if the there are poor interactions. For instance, the
ability to leak destructors forbids the original design for the scoped thread
API, and requires a few interfaces to do additional work. Alternative
standard libraries could provide completely different safety semantics to be
explored!
