\chapter{Rust}
\label{ch:rust}

Rust tries to be a practical language. It understands that each perspective has
advantages and drawbacks, and tries to pick the best tool for each job. From each
perspective, Rust takes one key insight. From the naive perspective, we embrace
that control is incredibly important. Rust wants to be used in all the places where
C and C++ are used today, and being able to manually manage memory is a critical
aspect of that. From the paranoid perspective, we embrace that completely
eliminating errors is great. Humans are fallible and get tired, but a good compiler
never rests and always has our back. From the suspicious perspective, we embrace
that compromises must be made for ergonomics.

Part of Rust's solution is then simply taking our favourite solutions for
each specific problem: static types, runtime bounds checks, no nulls, wrapping
arithmetic, and so on. However Rust has one very large holistic solution to
many problems, and it's what separates it from most other languages: ownership.
Rust's ownership model has two major aspects: controlling where and when data lives;
and controlling where and when mutation can occur. These aspects are governed by
three major features: affine types, regions, and privacy.






\section{Affine Types}

At a base level, Rust manages data ownership with an \emph{affine type system}. The
literature often describes affine types as being usable \emph{at most once}, but from
the perspective of ownership, affine typing means values are \emph{uniquely owned}.
To C++ developers, affine typing can be understood as a stricter version of
\emph{move semantics}.

If a variable stores a collection, passing it to a function by-value or
assigning it to another variable transfers ownership of the value to the new
location. The new location gains access to the value, and the old location loses
access. Whoever owns the value knows that it's the only location
in existence that can possibly talk about the contents of the collection.
This allows the owner of the collection to soundly \emph{trust} the
collection. Any properties it observes and wishes to rely on will not be changed
by some other piece of code without its permission. Perhaps more importantly,
the owner of the collection knows that it can do whatever it pleases with the
collection without interfering with anyone else.

A simple example:

\begin{minted}{rust}
fn main() {
    // A growable array
    let data = Vec::new();

    // transfer ownership of `data` to `data2`
    let data2 = data;
    // `data` is now statically inaccessible, and logically uninitialized

    // transfer ownership of `data2` to `consume`
    consume(data2)
    // `data2` is now statically inaccessible, and logically uninitialized
}

fn consume(mut data3: Vec<u32>) {
    // Mutating the collection is known to be safe, because `data3` knows
    // it's the only one who can access it.
    data3.push(1);
}
\end{minted}

The greatest of these rights is destruction: when a variable goes out of scope,
it destroys its value forever. This can mean simply forgetting the value,
or it can mean executing the type's destructor. In the case of a collection,
this would presumably recursively destroy all contained values, and free all of
its allocations.

Affine types are primarily useful for eliminating the \emph{use-after} family of bugs.
If only one location ever has access to a value, and a value is only invalidated
when that one location disappears, then it's trivially true that one cannot use
an invalidated value. For this reason, the most obvious applications of affine
typing are with various forms of transient resources: threads, connections,
files, allocations, and so on.

However it turns out that a surprising number of problems can be reduced to a
use-after problem. For instance, many APIs require some sequence of steps to
be executed in a certain order. This can be encoded quite easily using affine
types. Functions can produce a ``proof of work'' by returning a type that only
they have permission to produce. Similarly, functions can \emph{require} a proof
of work by consuming such a type:

\begin{minted}{rust}
fn first() -> First;
fn second(First) -> Second;
fn third(Second) -> Third;
fn alternative(First) -> Alternative;
\end{minted}

We can therefore use affine types to model valid control flow and statically
ensure correct usage. \emph{Session types} are the logical extreme of this
technique, where programs effectively ``write themselves'' due to their type
constraints. Munksgaard and Jespersen \cite{munksgaard2015practical} have an
excellent analysis of session typing in Rust, so we won't dwell on this topic.

It should be noted that affine typing isn't mandatory in Rust. Unique ownership
doesn't make sense or simply isn't important for many types like booleans and
integers. Such types can opt into \emph{copy semantics}. Copy types behave like any
other value with one simple caveat: when they're moved, the old copy of the
value is still valid.

Copy semantics can have surprising consequences though. For instance, it may be
reasonable for a random number generator to be copyable, as its internal
state is generally just some integers. It then becomes possible to
\emph{accidentally} copy the generator, causing the same number to be yielded
repeatedly. For this reason, some types which \emph{could} be copied safely don't opt
into copy semantics. In this case, affine typing is used as a lint against what
is likely, but not necessarily, a mistake.





\section{Borrows and Regions}

Affine types are all fine and good for some problems, but if that's all Rust had,
it would be a huge pain in the neck. In particular, it's very common to want
to \emph{borrow} a value. In the case of a unique borrow, affine types can encode
this fine: you simply pass the borrowed value in, and then return it back.

This is, at best, just annoying to do. Borrows must be ``threaded'' throughout
such an API explicitly in both the types and code:

\begin{minted}{rust}
struct Data { secret: u32 }

fn main() {
    let mut data = Data { secret: 0 }
    data = mutate(data);
    data = mutate(data);
    data = mutate(data);
}

fn mutate(mut data: Data) -> Data {
    data.secret += 1;
    return data;
}
\end{minted}

Affine types \emph{really} hit a wall when data wants to be \emph{shared}. If
several pieces of code wish to concurrently read some data, we have a serious
issue. One solution is to simply \emph{copy} the data to all the consumers. If each
has their own unique copy to work with, everyone's happy.

However, even if we're ignoring the performance aspect of this strategy (which
is non-trivial), it may simply not make sense. If the underlying resource to
share is truly affine, then there may be \emph{no} way to copy the data in a
semantic-preserving way. For instance, one cannot just blindly copy a file
handle, as each holder of the handle could then close it while the others are
trying to use it.

At the end of the day, having only values everywhere is just dang \emph{impractical}.
Rust is a practical language, so it uses a tried and true solution: pointers! Unfortunately,
pointers make everything more complicated and difficult. Affine types ``solved''
use-after errors for us, but pointers bring them back and make them \emph{far} worse.
The fact that data has been moved or destroyed says nothing of the state of
pointers to it. As C has demonstrated since its inception, pointers are all too
happy to let us view data that might be destroyed or otherwise invalid.

Garbage collection solves this problem for allocations, but does nothing to
prevent trying to use an otherwise invalidated value, such as a closed file.
Rust's solution to this problem is its most exotic tool: regions \cite{swamy2006safe}.

Like affine types, regions are something well-established in both theory and
implementation, but with little mainstream usage. Although Rust primarily cribs
them from Cyclone, they were first described by Tofte and Talpin \cite{tofte1997region}
and used in MLKit. That said, Cyclone's version of regions
is most immediately recognizable to a Rust programmer.

The idea of a region system is that pointers are associated with the region of
the program that they're valid for, and the compiler ensures that pointers don't
escape their region. This is done entirely at compile time, and has no runtime
component.

For Rust, these regions correspond to lexical scopes, which are roughly
pairs of matching braces. The restriction to lexical scopes is not fundamental,
and was simply easier to implement for the 1.0 release. It is however sufficient
for most purposes. Rust calls these regions \emph{lifetimes}.

At a base level, all a region system does is statically track what pointers are
outstanding during any given piece of code. By combining this information with
other static analysis it's possible to completely eliminate several classes of error
that are traditionally relegated to garbage collection. For ownership, region analysis allows us
to statically identify when a value is moved or destroyed while being pointed to,
and produce an error to that effect:

\begin{minted}{rust}
fn main() {
    // Gets a dangling pointer
    let data = compute_it(&0);
    println!("{}", data);
}

fn compute_it(input: &u32) -> &u32 {
    let data = input + 1;
    // Returning a pointer to a local variable
    return &data;
}
\end{minted}

\begin{minted}{text}
<anon>:11:13: 11:17 error: `data` does not live long enough
<anon>:11     return &data;
                      ^~~~
<anon>:8:36: 12:2 note: reference must be valid for the anonymous lifetime #1 defined on the block at 8:35...
<anon>: 8 fn compute_it(input: &u32) -> &u32 {
<anon>: 9     let data = input + 1;
<anon>:10     // Returning a pointer to a local variable
<anon>:11     return &data;
<anon>:12 }
<anon>:9:26: 12:2 note: ...but borrowed value is only valid for the block suffix following statement 0 at 9:25
<anon>: 9     let data = input + 1;
<anon>:10     // Returning a pointer to a local variable
<anon>:11     return &data;
<anon>:12 }
\end{minted}

On its own, this is pretty great: no dangling pointers without the need for
garbage collection! But when combined with affine types,
we get something even more powerful than garbage collection. For instance, if
you close a file in a garbage collected language, there is nothing to prevent
old users of the file from continuing to work with it. One must guard for
this at runtime. In Rust, this is simply not a concern:
it's statically impossible. Closing the file destroys it, and that means all
pointers must be gone. The problem of use-afters is once again solved.

Unfortunately, this doesn't solve problems like iterator invalidation. When an
iterator is invalidated, the collection it was pointing to wasn't \emph{destroyed},
it was just changed. In order to handle iterator invalidation, we require something
more than checking for moves.

The most extreme solution is to simply forbid internal pointers. Only allow
pointers to borrow variables on the stack, and everything else has to be copied
out. Then we never have to worry about pointers being invalidated. Unfortunately,
this would be a very limiting system. It would make composition of affine types useless,
because you could never access the components without destroying the aggregate.
It also doesn't really solve the problem the way we wanted. Most iterators we'd
be interested in providing would become inexpressible under this model. For
instance, one couldn't yield interior pointers from an iterator. Depending on
the details, any kind of tree iterator may be completely impractical.

Another extreme solution would be to forbid mutation of data. Mutations can be emulated
by creating a new object with the necessary changes, so this is in principle
possible. However this suffers from similar issues as borrow threading: It's
really annoying, and would also be make it difficult to obtain the same
control as C(++).

Yet another way is to treat all pointers into a collection as pointers \emph{to}
the collection, and forbid mutation through pointers. All mutating operations
could require by-value (and therefore unique) access, which could be done with borrow-threading.
This is unfortunate because we were trying to avoid borrow-threading by introducing
pointers in the first place, but at least we could share data immutably,
which is a definite win.

Rust basically takes this last approach, but in order to avoid the annoying pain of
threading borrows, it includes two different \emph{kinds} of pointer:
\emph{mutable references} and \emph{shared references} denoted \mintinline
{rust}{&mut} and \mintinline{rust}{&} respectively. Shared references are
exactly as we described: they can be
freely aliased, but only allow you to read the data they point to. On the other
hand, mutable must be unique, but enable mutation of the data
they point to. This means that taking a mutable reference to some data is like moving
it, but then having the compiler automatically insert all the boiler-plate
to move it back into place when the mutable reference is gone (the
compiler does not actually move the data around when you take a mutable
reference).

Let's look at some simple examples:

\begin{minted}{rust}
let mut data = 0;

{
    // Allowed to take multiple shared references
    let data_ref1 = &data;
    let data_ref2 = &data;

    // Allowed to read through them, and still read the value directly
    println!("{} {} {}", data_ref1, data_ref2, data);

    // Not allowed to mutate through them (compiler error)
    // *data_ref1 += 1;
}

{
    // Allowed to take one mutable reference
    let data_mut = &mut data;

    // Allowed to read or write through it
    println!("{}", data_mut);
    *data_mut += 1;

    // Allowed to move the mutable reference to someone else
    data_the_second = data_mut;

    // Not allowed to get an aliasing shared reference (compiler error)
    // let data_ref = &data;

    // Not allowed to get an aliasing mutable reference (compiler error)
    // let data_mut2 = &mut data;

    // Not allowed to directly access data anymore (compiler error)
    // println!("{}", data);
}

// All borrows out of scope, allowed to access data again
data += 1;
println!("{}", data);
\end{minted}



\section{Mutable XOR Shared}

This is Rust's most critical perspective on ownership: mutation is mutually
exclusive with sharing. In order to get the most out of this perspective, Rust
doesn't allow mutability to be declared at the type level. That is, a struct's
field cannot be declared to be constant. Instead, the mutability of a value is
\emph{inherited} from how it's accessed: as long as you have something by-value or
by-mutable-reference, you can mutate it.

This stands in contrast to the perspective that mutation is something to be
avoided completely. As we've seen, mutation can cause serious problems. This
has lead some to conclude that mutation should be avoided as much
as possible. Never mutating anything does indeed satisfy Rust's requirement
that sharing and mutating be exclusive, but in a vacuous way (mutating never
occurs). Rust takes a more permissive stance: mutate all you want as long
as you're not sharing. It is our belief that this eliminates most of the
problems that mutation causes in practice.

In particular, this statically eliminates iterator invalidation. For
instance, consider the following program:

\begin{minted}{rust}
fn main() {
    let mut data = vec![1, 2, 3, 4, 5, 6];
    for x in &data {
        data.push(2 * x);
    }
}
\end{minted}

What exactly the programmer intended here was unclear, and what exactly
will happen if this were allowed to compile is even more unclear.
Thankfully, in Rust we don't \emph{need} to wonder what the programmer meant or
what will happen when this is run, because it doesn't compile:

\begin{minted}{text}
<anon>:4:9: 4:13 error: cannot borrow `data` as mutable because it is also borrowed as immutable
<anon>:4         data.push(2 * x);
                 ^~~~
<anon>:3:15: 3:19 note: previous borrow of `data` occurs here; the immutable borrow prevents subsequent moves or mutable borrows of `data` until the borrow ends
<anon>:3     for x in &data {
                       ^~~~
<anon>:5:6: 5:6 note: previous borrow ends here
<anon>:3     for x in &data {
<anon>:4         data.push(2 * x);
<anon>:5     }
             ^
\end{minted}

This strategy also nicely generalizes to a concurrent context. Recall that a data race is
defined to occur when two threads access a piece of data in an unsynchronized
way, and one is writing. This is exactly aliasing and mutation, which is
forbidden by Rust's scheme. As such, everything in Rust is thread-safe by default.

Of course, perfectly good concurrent algorithms and data structures are rife with aliasing and
mutability. Mutexes exist \emph{precisely} to enable aliasing and mutation in a
controlled manner. As a result, although inherited mutability is the default way to do things in
Rust, it is not the only way. A few key types provide \emph{interior mutability},
which enables their data to be mutated through shared references as long as some
runtime mechanism ensures that access is properly restricted. The most
obvious example of this is exactly the standard library's Mutex type, which
allows an \mintinline{rust}{&Mutex<T>} to become an \mintinline{rust}{&mut T} by
acquiring its lock:

\begin{minted}{rust}
use std::sync::Mutex;

fn main() {
    // A Mutex owns the data it guards. In this case, an integer.
    // Note that `data` is not declared to be mutable, which normally
    // would make it impossible to update the value of the integer.
    let data = Mutex::new(0);

    {
        // Acquire the lock
        let mut handle = data.lock().unwrap();
        // A handle behaves like an `&mut` to the data
        *handle += 1;
        // But when it goes out of scope here, it releases the lock
    }
}
\end{minted}

Why is this interface sound? First and foremost, any attempt to acquire the
lock will block if it's already acquired. This ensures that only one handle
exists at any given time. However, we must also guarantee that the handle doesn't
outlive the mutex, and the pointer we get out of the handle doesn't outlive the
handle. These problems are handled by ownership. Affinity and region analysis
ensures that the pointers and handles aren't duplicated or allowed to outlive
the type they refer to.

However the entire reason we care about Mutexes is for sharing across threads.
This means there is one additional problem we must worry about: the shared
data not being thread-safe. As we have noted, almost everything in Rust is
actually thread-safe by default precisely because of ownership, but two things
can potentially break this: borrows, and interior mutability. Borrows aren't
trivially safe to share between two threads because they're based around
sequential scopes. They don't really make sense with concurrent executions.
Interior mutability isn't thread-safe because it's precisely sharing and mutation.

A Mutex provides interior mutability in an inherently thread-safe way, but not
all types do. In particular, the Cell and RefCell types \emph{aren't} thread-safe.
So how do we ensure that these problematic types aren't shared across threads?
For borrows, there's actually a way to declare that a type is expected to not
contain borrows, so anything that can pass data to another thread requires that.
However the interior mutability problem requires a completely different solution:
Traits.

Traits are Rust's version of an interface. Rust actually captures thread-safety
as traits that types can implement, called Send and Sync. If a type can be moved
to another thread safely, then it is Send. If a type can be shared between two
threads safely, then it is Sync. These traits are automatically derived
compositionally; if you consist entirely of Send types, then you are Send. This
works because of affinity and ownership. We know that if we own something
that is thread-safe, then we are the only ones who can access it. So if we're
accessed in a thread-safe way, then it's accessed in a thread-safe way.

Very few types are thread-unsafe, so almost everything is Send and Sync. However
some types are specifically thread-safe even though they're based on parts that
aren't. For instance, Mutex itself is based on parts that aren't thread-safe, but
it is of course thread-safe as long as it contains thread-safe data. As such,
types can manually claim to be Send or Sync. Of course, it's possible to
make this claim incorrectly, so how does is this safe to expose?

It's not safe. In fact, it's explicitly unsafe to implement these interfaces.



\section{Unsafe Rust}

Most languages are considered memory-safe. However with few exceptions, this
isn't actually true. In fact, basically \emph{every} language has unsafe bits.
The most fundamental of these is quite
simple: talking to C. C is the lingua-franca of the programming world. All
major operating systems and many major libraries primarily expose a C interface.
Any language that wants to integrate with these systems must therefore learn
how to talk to C. Because C is \emph{definitely} unsafe and can do just about anything
to your program, these languages then become transitively unsafe.

Rust is no different, but it embraces this reality a little more than most
other languages. Rust is actually \emph{two} languages: Safe Rust, and Unsafe Rust.
Safe Rust is the Rust we have been focusing on for the most part. It
is intended to be completely safe with one exception: it can talk to
Unsafe Rust. Unsafe Rust, on the other hand, is definitely not a safe language.
In addition to being able to talk to C (like any safe language), it enables the
programmer to work with several constructs that would be easily unsound in
Safe Rust. Most notably, for us, it allows Send and Sync to be implemented.
However Unsafe Rust is most commonly used because it includes raw C-like pointers
which are nullable and untracked.

At first glance, Unsafe Rust appears to completely undermine Rust's claims about
safety, but we argue that it in fact \emph{improves} its safety story. In most safe
languages, if one needs to do something very low level (for performance, correctness,
or any other reason) the general solution to this is ``use C''. This has several
downsides.

First, there's a cognitive overhead. Such an application now has
its logic spread across two completely different languages with different
semantics, runtimes, and behaviors. If the safe language is what a development
team primarily works in, it's unlikely that a significant percentage of the team
is qualified to actively maintain the C components. Second, it incurs non-trivial
runtime overhead. Data must often be reformatted at the language boundary, and
this boundary is usually an opaque box for either language's optimizer.
Finally, falling back to C is simply a \emph{huge} jump in unsafety, from ``totally
safe'' to ``pervasively unsafe''.

Unsafe Rust largely avoids these issues with one simple fact:
it's just a superset of Safe Rust. Lifetimes, Affine Types, and everything else
that helps you write good Rust programs are still working exactly as before.
You're just allowed to do a few extra things that are unsafe. As a result,
there's no unnecessary runtime or semantic overhead for using Unsafe Rust.

Of course one \emph{does} need to understand how to manually uphold Safe Rust's
various guarantees when using Unsafe Rust's extra parts, and this isn't trivial.
However this is still a better situation than using C, because the unsafety is
generally much more modular. For instance, if you use Unsafe Rust to index into
an array in an unchecked manner, you don't suddenly need to worry about the
array being null, dangling, or containing uninitialized memory. All you need to
worry about is if the index is actually in bounds. You know everything else is
still normal.

In addition, Unsafe Rust doesn't require any kind of complicated foreign
function interface. It can be written inline with Safe Rust on demand. Rust's
only requirement is that you write the word ``unsafe'' \emph{somewhere} to indicate
that you understand that what you're doing is unsafe. Since unsafety is
explicitly denoted in this manner, it also enables it to be detected and linted
against if desired.

Rust's standard library (which is written entirely in Rust) makes copious use of
Unsafe Rust internally. Most fundamentally, Unsafe Rust is necessary to provide
various operating system APIs because those are written in C, and only Unsafe
Rust can talk to C. However Unsafe Rust is also used in various places to
implement core abstractions like mutexes and growable arrays.

It's important to note that the fact that these APIs use unsafe code is entirely
an implementation detail to those using the standard library. All the unsafety
is wrapped up in \emph{safe abstractions}. These abstractions serve two masters: the
consumer of the API, and the producer of the API. The benefit to consumers of an
API is fairly straight-forward: they can rest easy knowing that if something
terrible happens, it wasn't their fault. For producers of the API, these safe
abstractions mark a clear boundary for the unsafety they need to worry about.

Unsafe code can be quite difficult because it often relies on stateful
invariants. For instance, the capacity of a growable array is a piece of state
that unsafe code must trust. In order to be sound, these safe abstractions need
to rely on the final element of ownership: \emph{privacy}.

Privacy in Rust is much the same as in most other languages. Fields and functions may be marked
as public or private, and only code that is within some boundary may access anything that is
marked private.

Returning to our example, the capacity of an growable array is marked as private.
Since the abstraction boundary is often exactly the privacy boundary in Rust,
end users of a growable array are therefore prevented from directly manipulating
the capacity. Within the array's privacy boundary, this state can be arbitrarily
manipulated, but this is a closed set of code to audit and verify. The code within
the privacy boundary can therefore trust that the capacity field is only updated
by a small set of trusted code.

This rest of this thesis focuses primarily on these safe abstractions. A good safe abstraction
must have many properties:

* Safety: Using the abstraction inappropriately cannot violate Rust's safety guarantees.
* Efficiency: Ideally, an abstraction is \emph{zero cost}, meaning it is as efficient
  at the task it is designed to solve as an unabstracted solution (with a decent
  optimizing compiler).
* Usability: A good abstraction should be more convenient and easy to understand than
  the code it's wrapping.

It would be \emph{excellent} if the implementation was also completely safe, but we
do not consider this a critical requirement, as Rust's standard library demonstrates.

It should be noted that Rust's reliance on safe abstractions is, in some sense,
unfortunate. For one, it makes reasoning about the performance characteristics
of a program much more difficult, as it relies on a sufficiently
smart compiler to tear away these abstractions. This in turn means Rust's unoptimized
performance in a rather atrocious state. It's not uncommon for a newcomer to
the language to express shock that a Rust program is several times slower than
an equivalent Python program, only to learn that enabling optimizations makes
the Rust program several times \emph{faster} than the Python program.
However it is our opinion that this is simply fundamental to providing a
programming environment that is safe, efficient, and usable.


