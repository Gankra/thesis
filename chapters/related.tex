\chapter{Related Work}
\label{ch:related}

Since Rust's model of ownership is built up of well-established parts, it should
surprise no one to learn that other languages have been built with similar
designs. Of particular interest are C++ and Cyclone, the two languages closest
to Rust.




\section{Cyclone}

The Cyclone programming language is easily the closest relative to Rust. Cyclone
was a project with similar goals to Rust: make a language that can do all the
low-level stuff normally reserved for C and C++, but make it safe. However, while
Rust was designed as a completely new language, Cyclone was designed to be a
minimal addition on top of C. The ideal of Cyclone was that existing C code
could be migrated to it relatively painlessly. This ultimately limited
Cyclone's semantics, resulting in a system that was more awkward to use than
Rust.

The Cyclone project was officially abandoned with its 1.0 release in 2006.
Its official website links only to Rust for those interested in Cyclone's ideas
\cite{cyclone-web}. Rust's official website in turn cites Cyclone as an
influence for region analysis \cite{rust-stolen}.

Early versions of Cyclone did not include affinity at all, as all types in C
are always copyable. Cyclone also initially used region analysis primarily to
manage dynamic pools of heap allocations, in much the same way that Tofte and Talpin's
MLKit work did. In order to be sound, these early versions had to make C's $free$ function
a no-op, leaking the memory \cite{grossman2002region}.

However the final release of Cyclone introduced several constructs
that made it quite close to Rust 1.0 \cite{swamy2006safe}. In particular, Cyclone introduced
a \emph{unique pointer} type which could be manually allocated and freed. In order
to avoid use-after-frees, these pointers had to be affine. Internal references to a
unique pointer could be taken, and were tracked in much the same way internal references are
tracked in Rust. Unfortunately, no general mechanism was exposed for making arbitrary
values or types affine, making this only usable for managing allocations.
Since the project was officially abandoned after the release of unique pointers, we
do not believe that affinity was explored in any further depth in Cyclone.






\section{C++}

C++ is the 8000-pound gorilla of systems programming in more ways than one. Not
only is it one of the two major choices for projects that want low-level control,
it's also incredibly complicated. Just like Cyclone, an initial
desire to be a drop-in replacement for C resulted in significantly more complicated
ownership semantics.

C++'s initial solution to the problem of pervasive copying in C was to introduce
the ability to overload the behaviour of assignment and copying. In C if a value
semantically owns an allocated buffer, copying that value will create two values
with the the same buffer pointer, which isn't correct. In C++, that type would
probably overload copying to create a new allocation for the copy. This does
indeed solve the problem, but introduces several new problems.

The ability for every assignment and copy to execute arbitrary code makes it
much more difficult to reason about correctness and performance of a piece of
code. For instance, the statement `x = y` could involve allocating and copying
a massive buffer. In addition, this creates a serious exception-safety problem,
as now \emph{everything} can throw an exception.

For these reasons, Rust has specifically avoided overloading copying and assignment,
defining both to just always be a bitwise copy. This is possible in Rust because values
that own resources are affine, so the compiler will appropriately mark them
as unusable, preventing the duplicated pointer problem.

To avoid the cost of expensive copies, C++11 introduced \emph{move semantics}.
Moving a value to a new location allows the old location to be cannibalized for
the resources it owns, with the new location taking ownership of them. This is
the same basic idea as affine typing in Rust, but with two major differences: the
old location must still contain a valid instance, and arbitrary code can be
executed to complete the move. Once again, these constraints leave C++'s solution
significantly more difficult to reason about and use correctly.

C++'s unique\_ptr type is a pointer to an owned value on the heap. When moved,
the old location's value is set to 0, indicating that there is no allocation.
Dereferencing such a pointer is Undefined Behaviour, but because the value is
``valid'', C++ will do nothing to prevent someone from doing this:

\begin{minted}{C++}
#include <iostream>
#include <memory>

int main() {
    // x owns the allocation
    auto x = std::make_unique<int>(0);
    // now y owns the allocation
    auto y = std::move(x);
    // dereference null x; oops!
    std::cout << *x;
    std::cout << *y;
}
\end{minted}

Even when passed the most pedantic warning flags, the latest version of clang
happily compiles this program without any indication that something bad could
happen. The equivalent Rust program fails to compile:

\begin{minted}{rust}
fn main() {
    let x = Box::new(0);
    let y = x;
    println!("{}", x);
    println!("{}", y);
}
\end{minted}

\begin{minted}{text}
<anon>:4:20: 4:21 error: use of moved value: `x` [E0382]
<anon>:4     println!("{}", x);
                            ^
\end{minted}

It can also do less work, because x does not need to be overwritten in
Rust. The fact that x has been moved out of can almost always be statically
tracked.

It should be noted that C++'s solution does have one explicit advantage: it
can encode types which have significant locations. For instance, one could
construct a linked list with nodes stored on the stack in C++, and have the
move constructor for a node update the pointers of its neighbors. In Rust
this design would never be sound, because a value isn't ever informed when it's
moved. However it's the Rust developer's opinion that this is a relatively
minor loss compared to the other benefits of Rust's solution.

C++ has no solution to the problem of borrowed pointers. Any interface that
exposes an untracked pointer or reference is fundamentally unsafe, because they
can be made to dangle. If C++ code wishes to be safer, this necessarily implies
copying or moving the data, which is often less efficient than passing a reference
(especially because copies and moves invoke arbitrary code).

However Bjarne Stroustrup, the creator of C++, is currently working on a project
to resolve this issue. The \emph{C++ Core Guidelines} \cite{cppcore} are an attempt to extend
C++ with some new semantics, and then identify a safe subset which can be compiler
verified. This project was announced only a few months ago, so little analysis of the
system exists.

Our understanding is that rather than using the well-established system of
region analysis, it's based on a custom system that maintains lists of what
points to what. For simple cases it seems to behave in much the
same way as Rust's system, and requires similar annotations. However we have been
unable to get in touch with the developers of the system to discuss the details,
which are unclear from the publicly available details. Only time will tell
what becomes of this experiment.








% \section{Alms}

% Alms has affine types, but exposes a more advanced mechanism than regions. It's
% sufficient to implement regions as a library notion.



% TODO?
% \section{Vault}

% Vault uses linear typing instead of affine typing.



