\chapter{Introduction}
\label{ch:intro}

Modern systems are built upon shaky foundations. Core pieces of computing
infrastructure like kernels, system utilities, web browsers, and game engines
are almost invariably written in C and C++. Although these languages provide
their users the control necessary to satisfy their correctness and performance
requirements, this control comes at the cost of an overwhelming burden for the
user: \emph{Undefined Behaviour}. Users of these languages must be ever-vigilant
of Undefined Behaviour, and even well-maintained projects fail to do so
\cite{wang2012undefined}. The price for invoking Undefined Behaviour is dire:
compilers are allowed to do \emph{absolutely anything at all}.

Modern compilers can and will analyze programs on the assumption that Undefined
Behaviour is impossible, leading to counter-intuitive results like time travel,
wherein correct code is removed because subsequent code is Undefined. This
aggressive misoptimization can turn relatively innocuous bugs like integer
overflow in a print statement into a wildly incorrect behaviour. Severe
vulnerabilities subsequently occur because Undefined Behaviour often leads to
the program blindly stomping through memory and bypassing checks, providing attackers
great flexibility.

As a result, our core infrastructure is constantly broken and exploited. Worse,
C and C++ are holding back systems from being as efficient as they could be.
These languages were fundamentally designed for single-threaded programming and
reasoning, having only had concurrency grafted onto their semantics in the last
5 years. As such, legacy code bases struggle to take advantage of hardware
parallelism. Even in a single-threaded context,
the danger of Undefined Behaviour encourages programs to be written
inefficiently to defend against common errors. For instance it's much safer
to copy a buffer than have several disjoint locations share pointers to it,
because unmanaged pointers in C and C++ are wildly unsafe.

In order to address these problems, Mozilla developed the Rust programming language.
Rust has ambitious goals. It intends to be more efficient than C++ and safer
than Java without sacrificing ergonomics. Although Rust 1.0 was released less
than a year ago \cite{rust1}, early results are incredibly promising. To see this, we need
look no further than the Servo project, a rewrite of Firefox's core engine in Rust
\cite{servo}.

Preliminary results \cite{servo-exp} have found that Servo can
perform layout two times faster than Firefox on a single thread, and four times
faster with four threads. Servo has also attracted over 300 contributors \cite{servo-gh},
in spite of the fact that it's written in a language that was barely just
stabilized, demonstrating the maintainability of Rust applications. There's
little data on safety and correctness in Servo, but we will see throughout this
thesis that Rust provides powerful tools for ensuring safety and correctness.

It should be noted that Rust doesn't desire to invent novel systems or analyses,
because research is hard, slow, and risky. Indeed, to a well-versed programming
language theorist Rust can be easily summarized as a list of well-studied
ideas. However, Rust's type system as a whole is greater than the sum
of the parts. As such, we believe it merits research.

Truly understanding Rust's type system in a rigorous way is a massive
under-taking that will fill several PhD theses \cite{rustbelt}. As such, we will make no effort
to do this. Instead, we will focus in on what we consider the most interesting
aspect of Rust: \emph{ownership}. Ownership is an emergent system built on three
major pieces: \emph{affine types}, \emph{region analysis}, and \emph{privacy}. Together we get
a system for programmers to manage where and when data lives, and where
and when it can be mutated in a reasonably ergonomic and intuitive manner.

In particular,
we observe that many of the kinds of problems Rust seeks to solve boil down
to issues of \emph{trust} between different pieces of code and the data they share.
Because Rust's ownership system allows developers to model access to data in
novel ways, it's possible to cleanly model the trust these interfaces require to be
safe and efficient.

Ownership particularly shines when one wants to develop concurrent programs,
an infamously difficult problem. Using Rust's tools for generic programming,
it's possible to build totally safe concurrent algorithms and data structures
which work with arbitrary data without relying on any particular paradigm like
message-passing or persistence. Rust's crowning jewel in this regard is the
ability to safely have child threads mutate data on a parent thread's stack
without any unnecessary synchronization, as the following program demonstrates. (If you
don't understand Rust syntax, a brief introduction is provided in the Appendix)

\begin{minted}{rust}
[dependencies]
crossbeam = 0.1.6

-----

extern crate crossbeam;

fn main() {
    let mut array = [1, 2, 3];

    // Create a scope to know when to join all threads
    crossbeam::scope(|scope| {

        // Get pointers to each element in the array
        for x in &mut array {
            // Spawn a thread to increment this element of the array
            scope.spawn(move || {
                *x += 1;
            });
        }

        // Block on all the child threads joining here
    });

    println!("{:?}", array);
}
\end{minted}

The most amazing part of this program is that it's based entirely on a third-
party library (crossbeam \cite{crossbeam}). Rust does not require threads to be modeled as part
of the language or standard library in order for these powerful safe
abstractions to be constructed. Any attempt to share non-threadsafe data with
the scoped threads, give the same pointer to two threads, keep a pointer for too
long, or access the array while the threads are running won't just fail to work,
it will \emph{fail to compile}. The fact that misuse is a static error is incredibly
important for concurrent programs, because runtime errors can be incredibly
difficult to reproduce or debug.

Before we get into all the finer details of how this example works, we'll need
to build up some foundations. In chapter 2, we establish
the basic principles of safety and correctness that we're interested in. Since
these are ultimately vague concepts in a practical setting, we do this by
surveying several classic errors that all languages must deal with. In chapter 3
we identify how each kind of error can be understood in terms of
\emph{trust} and the major strategies for handling trust issues. In chapter 4
we give a brief introduction to Rust's syntax, semantics, and ownership. In
chapter 5 we analyze several interfaces designed by ourselves and the greater
Rust community in terms of trust. In chapter 6 we we examine the limitations of
ownership. Finally, in chapter 7 we briefly survey how ownership in Rust compares
to similar systems in other languages.
