\chapter{Trust}
\label{ch:trust}

We believe that the three memory-safety errors we described in the previous
section, and many more like them, all boil down to a single issue: data trust.
All programs require or expect data to have certain properties. This trust
is so pervasive that it's easy to forget that we're even making assumptions about
how data will behave. This trust has many aspects:




First, there is the \emph{property} of the data that is trusted. The simplest form of
trust is trusting that certain values don't occur: booleans are either `true`
or `false`, but never `FileNotFound`. This kind of property is generally quite
manageable because it's easy to verify on demand.

The more complex form of property is trusting that separate pieces of data agree.
This is the kind of property that we see in our memory-safety errors. Indices need
to agree with the length of the array, the allocator needs to agree that
dereferenced pointers are allocated, and iterators need to agree with the collection
they iterate. This is the kind of trust we will be focusing on for the bulk of
this work.




Second, there is the \emph{who} to trust with data. For most applications, it's pretty
reasonable to design an interface that assumes that the language, hardware, and
operating system are well-behaved. Of course, bugs in any of these systems
may necessitate workarounds, but we don't consider this aspect to be particularly
interesting. At some limit, these systems just work the way they work, and any
oddities can be regarded as a nasty corner-case to be documented.

The aspect of who to trust that we consider really interesting is within the
boundary of userland. Within a single program, functions need to decide whether
to trust other functions. It's fairly reasonable to trust a single concrete
function to be well-behaved. For instance, if one wishes to invoke a print
function that claims to only read its inputs, it's reasonable to rely on that
fact. We consider it reasonable to extend our trust here because it's being done
in a \emph{closed} manner. That is, it's in principle possible to test or otherwise
verify that the print function doesn't mutate its input, especially if the
source is available.

The problematic case is when we need to extend our trust in an \emph{open} manner. Public
functions can be called by arbitrary code, and as such need to worry about trusting
their caller. Do we trust the caller to pass us correct inputs? Do we trust the caller
to invoke us at the right time? Similarly, generic or higher-order functions need to
worry about trusting the arbitrary functions that they're given. If we were \emph{given}
a printing function instead of getting to call some concrete
one, we then have to worry if that print function will uphold its promise not to mutate
the input.




Third, there is the \emph{consequences} of the trust; what happens if trust
is violated? Consequences for an application can range from marginal to catastrophic;
it could perform slightly worse, produce nonsensical output, crash, delete the
production database, or compromise the system it's running on.
The consequences of an assumption can easily shape whether that assumption is
made at all. Certainly, leaving the system vulnerable is not something to be
taken lightly.




Fourth and finally, there is the \emph{justification} of the trust. At one extreme, one may simply
blindly trust the data. Indexing is unchecked; pointers are assumed to be
allocated; and booleans are assumed to be true or false.

At the other extreme, one can statically guarantee that data is correct. Static
type checking can help ensure that a boolean doesn't contain FileNotFound,
control flow analysis can determine if a pointer is used after it's freed, and
normalized databases ensure that there is a single source of truth for each fact.

Somewhere in between these two extremes, one can \emph{validate} or \emph{correct} data
at runtime. Indexes may be checked against an array's length; inputs may
be escaped; and iterators can check that their collection hasn't been modified.

We respectively call these three approaches \emph{naivety}, \emph{paranoia}, and \emph{suspicion}.
As we will see, they each have their own strengths and weaknesses that makes them appropriate
for different tasks. As such, basically every piece of software relies on a combination
of these strategies to strike a balance between the competing demands of control, safety,
and ergonomics. That said, some systems clearly favour some strategies more than others.



\section{Naivety}

A naive interface isn't a fundamentally bad interface. It's just one that trusts
that the world is great and everyone is well-behaved. Providing a naive interface
is a bit like giving everyone you meet a loaded gun and trusting that they use
it wisely. If everyone \emph{is} well-behaved, then everything goes great. Otherwise,
someone might lose a leg.

First and foremost, a naive interface
is simply the easiest to \emph{implement}. The implementor doesn't need to concern
themselves with corner cases because they are simply assumed to not occur. This
in turn gives significant control to the user of the interface, because they don't
need to worry about the interface checking or mangling inputs. It's up to the user
to figure out how correct usage is ensured. Since the user has the most context,
they are the best poised to identify when assumptions can or can't be made.

Naive interfaces may also expose lower-level details more freely, on the
assumption that they will be used correctly. This empowers users to effeciently
and reliably perform operations that may never have been envisioned by the
original implementor, or were simply deemed too much work to develop and maintain.
Control is absolutely the greatest strength of naive interfaces. They empower
excellent programmers to produce excellent code where it matters.

For similar reasons, naive interfaces are also, in principle, ergonomic to work
with. In particular one is allowed to try to use the interface in whatever
manner they please. This enables programs to be transformed in a more continuous
manner. Intermediate designs may be incorrect in various ways, but if one is
only interested in exploring a certain aspect of the program that doesn't depend
on the incorrect parts, they are free to do so.

The uncontested champion of this approach is C. In particular, its notion of
\emph{Undefined Behaviour} is exactly naivety. C programs are expected to uphold
several non-trivial properties, and the C standard simply declares that if those
properties \emph{don't} hold, the program is \emph{undefined}. This naivety can be found in
relatively small and local details like unchecked indexing, as well as massive
and pervasive ones like manual memory management.

Undefined Behaviour demonstrates the extreme drawback of naive
interfaces. Breaking C's trust has dire consequences, as the compiler is free to misoptimize the
program in arbitrary ways, leading to memory corruption bugs and high severity
vulnerabilities. Modern compilers try to give
a helping hand with debug assertions and static analysis, but blind trust is
evidently too deeply ingrained into C. Programs written in languages like C
continue to be exploited, and no end to this is in sight.

If the consequences of misuse are high or the contracts that must be upheld are
too onerous, the ergonomic benefits of naive interfaces can be undermined.
Having to be always vigilant against incorrect usage can be mentally exhausting.
The problems that result from misusing a naive interface can also be cryptic and
difficult to debug, due to the fact that they don't tend to fail fast, but instead
silently corrupt the system.

That said, the control that C's trusting design provides is highly coveted.
Operating systems, web browsers, game engines, and other widely used pieces of
software continue to be primarily developed in C(++) because of the perceived
benefits of this control. Performance is perhaps the most well-understood benefit
of control, but it can also be necessary for correctness in the case of hardware
interfaces. Languages without this control are simply irrelevant for many projects.





\section{Paranoia}

Paranoid designs represent a distrust so deep that one believes things must be
taken away so as to eliminate a problem \emph{a priori}. Where the naive would give
everyone a loaded gun, the paranoid are so concerned with people getting shot
that they would try to make it impossible by eliminating bullets, guns, or even
people.

There are two major families of paranoid practices: omissions and static analysis.

Omissions represent pieces of functionality that could have been provided or
used, but were deemed too error-prone. The much-maligned `goto` is perhaps
the most obvious instance of a paranoid omission, but it's not the most
interesting. The most interesting omission is the ability to manually free memory,
which is the omission that all garbage collected languages effectively make.
Some less popular examples include always-nullable pointers (Haskell, Swift, Rust),
concurrency (JavaScript), and side-effects (pure functional programming).

Static analysis consists of verifying a program has certain properties at
compile-time. Static typing is the most well-accepted of these practices,
requiring every piece of data in the program to have a statically known type.
Various static lints for dubious behaviours also fall into this category,
including code style checks. This analysis also includes more exotic systems
like dependent-typing, which expresses requirements on runtime values (such as
`x < y`) at the type level.

The primary benefit of paranoid practices is maximum safety. Ideally, a paranoid
solution doesn't just handle a problem, it \emph{eliminates} that problem. For
instance, taking away memory freeing completely eliminates the use-after-free,
because there is \emph{no} ``after free'' (garbage collection being ``only'' an
optimization). Similarly, a lack of concurrency eliminates data races, because
it's impossible to race. Lints may completely eliminate specific mistakes, or
simply reduce their probability by catching the common or obvious instances.
Dependent typing can eliminate indexing out of bounds, by statically proving
that all indices are in bounds \cite{xi1998eliminating}.

The cost of this safety is usually control or ergonomics. Paranoid practices point to the pervasive
problems that unchecked control leads to, and declare that this is why we can't have nice
things. Certain operations must be forbidden, or require burdensome annotation.
It's worth noting that a loss of control does not necessarily imply a
loss in performance. In fact, quite the opposite can be true: an optimizing
compiler can use the fact that certain things are impossible to produce more
efficient programs \cite{coutts2007stream}.

That said, paranoid practices \emph{do} generally take away a programmer's ability to
directly construct an optimized implementation. One must instead carefully craft
their program so that it happens to be optimized in the desired manner. This
leaves the programmer dependent on the compiler to understand how to produce the
desired idiom, and may require significant annotations to convince it that this
is desirable. In some cases, this means that the desired output is simply
impossible.

The ergonomic impact of paranoid practices is a tricky subject. For the most
part, paranoid practices err on the side of producing false negatives -- better to
eliminate some good programs than allow bad programs to exist! This can lead to
frustration if one wishes to produce a program they believe to be a false
negative.

Omissions are generally much more well-accepted in this regard, particularly
when an alternative is provided that handles common cases. The
absence of manual free is quite popular because freeing memory is purely annoying
book-keeping. Freeing memory has no inherent semantic value, it's just necessary
to avoid exhausting system resources. The fact that garbage collection just
magically frees memory for you is pretty popular, as evidenced by its dominance
of the programming language landscape (C and C++ being the most notable exceptions).
However some omissions can be too large to bear. Few systems are designed in
a pure-functional manner.

Static analysis is much more maligned, as the developers of coverity found
\cite{bessey2010few}. It's easy to forgive an omission, particularly if one isn't
even aware that an omission occurred (how many programmers are even aware of
goto these days?). Static analysis, on the other hand, is active and in the programmer's face.
It continuously nags the programmer to do a better job, or even prevents the
program from running. Static typing, one of the less controversial
strategies, is rejected by many popular languages (JavaScript, Python, Ruby).
Meanwhile, dependent types struggle to find any mainstream usage at all. These practices
can also harm the ability to continuously transform a program, forcing the
programmer to prove that \emph{everything} is correct in order to test \emph{anything}.

Regardless, embracing paranoia \emph{can} be liberating. Once a paranoid
practice is in place, one never needs to worry about the problems it addresses
again. Once you start collecting your own rainwater, why worry what's being put
in the water supply?




\section{Suspicion}

Finally we have the position of compromise: the suspicious practices. A suspicious
design often looks and acts a lot like a trusting design, but with one caveat: it
actually checks at runtime. The suspicious stance is that some baseline level of
safety is important, but categorically eliminating an error with a paranoid practice
is impractical. The suspicious might give everyone guns, but make them explode whenever
a HumanPointerException is detected.

Suspicious practices can skew to be implicit or explicit, in an
attempt to fine-tune the tradeoffs.

In languages where pointers are nullable by default, it's common for all dereferences
to silently and implicitly check for null. This may cause the program to completely
crash or trigger unwinding (an ``exception'' in many languages). This allows the programmer
to write the program in a trusting way without worrying about the associated danger.

Other interfaces may require the author to explicitly acknowledge invalid cases.
For instance, an API may only provide access to its data through callbacks which it
invokes when the data is specifically correct or incorrect. This approach is often
more general, as it can be wrapped to provide the implicit form. Compared to
implicit checks, explicit checks give more control to the user and can improve the
clarity of code by making failure conditions more clear. But this comes at
a significant ergonomic cost; imagine if every pointer dereference in Java
required an error callback!

Regardless of the approach, suspicious practices are a decent compromise. For
many problems the suspicious solution is much easier to implement, work with,
and understand than the paranoid solution. Array indexing is perhaps the best
example of this. It's incredibly simple to have array indexing unconditionally
check the input against the length of the array. Meanwhile, the obvious paranoid
solution to this problem requires an integer theorem solver and may require
significant programmer annotations for complex access patterns. That said,
paranoid solutions can also be simpler. Why check every dereference for
null, when one can simply not let pointers be null in the first place?

One significant drawback of suspicious practices, particularly of the implicit
variety, is that they do little for program \emph{correctness}. They just make sure
that the program doesn't do arbitrary unpredictable things with bad data, usually by
reliably crashing the program. This can be great for detecting, reproducing,
and fixing bugs, but a crash is still a crash to the end user. This can be
especially problematic for critical systems, where a crash is essentially a
critical vulnerability.

Suspicious solutions also often have the worst-in-class performance properties
of the three solutions. Trusting solutions let the programmer do whatever they
please while still aggressively optimizing. Paranoid solutions give the
compiler and programmer rich information to optimize the program with.
But suspicious solution gets to assume little, and must bloat up
the program with runtime checks which further inhibit other optimizations.
That said, a good optimizing compiler can completely eliminate this gap in
some cases. Bounds checks in particular are easily eliminated in common cases
\cite{grosser2011polly}.




