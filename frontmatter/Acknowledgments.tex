%*******************************************************
% Acknowledgments
%*******************************************************
%\pdfbookmark[1]{Acknowledgments}{acknowledgments}
\phantomsection
\addcontentsline{toc}{chapter}{\tocEntry{Acknowledgments}}

% \begin{flushright}{\slshape
%     We have seen that computer programming is an art, \\
%     because it applies accumulated knowledge to the world, \\
%     because it requires skill and ingenuity, and especially \\
%     because it produces objects of beauty.} \\ \medskip
%     --- \defcitealias{knuth:1974}{Donald E. Knuth}\citetalias{knuth:1974} \citep{knuth:1974}
% \end{flushright}

\bigskip

\begingroup
\let\clearpage\relax
\let\cleardoublepage\relax
\let\cleardoublepage\relax
\chapter*{Acknowledgments}

This work wouldn't have been possible without the amazing help from everyone in
the Rust community.

I'd first and foremost like to thank Aaron Turon, who is perhaps the greatest
editor ever, in addition to being an amazing academic, programmer, and boss.
Over the past two years, Aaron has repeatedly pushed me to produce better content
with his insightful and accurate analysis. He also somehow managed to imbue me with
a good chunk of PL theory and type theory, in spite of my initial dismissal of these
areas.

Additional thanks must be given to Niko Matsakis, Alex Crichton, Brian Anderson,
Felix Klock, Nick Cameron, Huon Wilson, Patrick Walton, Steve Klabnik, Ariel Ben-Yehuda,
James Miller, Björn Steinbrink, Eduard Burtescu, Ulrik Sverdrup, and so many more members of the
Rust community for helping me understand the ever-illusive problem of
\emph{making computers do things good}.

I'd also like to thank Andrew Paseltiner, Jonathan ``gereeter'' S., Piotr Czarnecki,
Clark Gabel, and Jonathan Reem for helping out so much with Rust's collections
libraries. Andrew in particular for keeping everything up to date while I've been
busy writing all this text.

My actual supervisors Michiel Smid and Pat Morin deserve serious respect for
dealing with the 4 different thesis topics I have produced over the last
two years. Particularly as each subsequent thesis attempt has been increasingly
unrelated to the problems that they are actually experts in.

Finally, my sister Julia did all of the dishes. So many dishes.


\endgroup



