%*******************************************************
% Abstract
%*******************************************************
%\renewcommand{\abstractname}{Abstract}
%\pdfbookmark[1]{Abstract}{Abstract}
\phantomsection
\addcontentsline{toc}{chapter}{\tocEntry{Abstract}}
\begingroup
\let\clearpage\relax
\let\cleardoublepage\relax
\let\cleardoublepage\relax

\chapter*{Abstract}
Rust is a new programming language developed by Mozilla in response to the fact
that C and C++ are unsafe, inefficient, and unergonomic -- particularly when
applied to concurrency. Version 1.0 of Rust was released in May 2015, and appears
to be performing excellently. Rust code is memory-safe
by default, faster than C++, easier to maintain, and excels at concurrency.

Yet little analysis exists of the semantics and expressiveness of Rust's type
system. This thesis focuses on one of the core aspects of Rust's type system:
\emph{ownership}. Ownership is a system for expressing where and when data lives, and
where and when data can be mutated. In order to understand ownership and the
problems it solves, we provide a novel analysis of both in terms of \emph{trust}.

We observe that many classical memory safety errors are the result of some
code trusting data to be in a particular state that doesn't necessarily hold.
For instance, indexing out of bounds occurs when one incorrectly trusts an
index to agree with an array's length. The problem of trust is largely addressed using three
different strategies: \emph{naivety}, \emph{paranoia}, and \emph{suspicion}. Each
of these strategies represents a compromise among the competing concerns of control, safety, and
ergonomics. Briefly, naive interfaces optimize for control by assuming that
state is correct; paranoid interfaces optimize for safety by statically preventing
invalid states; and suspicious interfaces optimize for ergonomics by validating
states at runtime.

We demonstrate how ownership can be used to build efficient and ergonomic
paranoid interfaces, even when building on top of unsafe naive interfaces. For example,
iterators in Rust provide an interface for indexing into an array
without bounds checks while being statically immune to invalidation.
Two of these interfaces, \emph{drain} and \emph{entry}, were developed by us for Rust's
standard library.

Finally we demonstrate some of the limits of ownership. In particular, we
explore how the lack of proper linear typing makes it difficult to guarantee
that desirable operations are performed, such as freeing unused memory or
repairing invariants. However we provide some tricks for \emph{ensuring} that operations
are performed (at the cost of ergonomics), or simply mitigating the impact of not
performing them.

\endgroup

\vfill
